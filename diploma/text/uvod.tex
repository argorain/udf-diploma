\chapter*{Úvod}
\phantomsection
\addcontentsline{toc}{chapter}{Úvod}

Diagnostika integrity dat je klíčovou částí správy a uchovávání dat. Bez ní nemáme žádnou informaci o jejich konzistenci ani o jejich správnosti a vzhledem k faktu, že data jsou nejcennější část výpočetní techniky, je potřeba této problematice věnovat dostatečnou pozornost.

Byla to právě data co stálo na začátku zrodu výpočetní techniky. Bylo jich stále více a lidské zdroje přestávaly na narůstající objem stačit. Původní úložné médium byla papírová karta ve složce v kartotéce a ta byla seřazena podle určitého řádu. Za ekvivalent dnešních metadat souborového systému by bylo možné považovat popisné štítky složek a zásuvek a rešeršní katalogy. Pokud došlo k chybě při založení složky, šance, že na to někdo přijde dřív, než ji bude potřebovat byla prakticky nulová. Jinými slovy, diagnostika a kontrola integrity dat neexistovala, předpokládalo se, že systém je natolik snadný, že není potřeba. Ovšem ve chvíli, kdy psaný text nahradil kód, nejprve ve formě otvorů na páskách a štítcích a později magnetické záznamy, člověk už nedokázal případnou zatoulanou složku ani poznat, natož opravit její zařazení. A tady začíná pole působnosti nástrojů kontrolující integritu dat.
 
Cílem této práce je vytvořit nástroj pro diagnostiku konzistence dat souborového systému v \mbox{GNU/Linux}, pro který tyto nástroje neexistují. Bude provedena rešerše existujících souborových systémů a jejich diagnostických nástrojů a vybrán dostatečně perspektivní souborový systém, kterému tyto nástroje chybí. Dostatečně perspektivním je myšlen souborový systém, který má i v dnešní době využití, ať už z historických důvodů nebo z důvodu popularity. Perspektivní by rozhodně nebylo dopisovat chybějící nástroje pro souborový systém nasazený například pro historický operačná systém \emph{Unix System V}.

Různé souborové systémy jsou určeny přednostně pro určitá fyzická média. Tomu odpovídá i jejich design, který obvykle respektuje silné a slabé stránky daného média. To také odráží i druh chyb, které se na daném médiu vyskytují a to, jak se s nimi vypořádává souborový systém. Lze předpokládat, že jinými ochrannými mechanismy bude opatřeno optické médium, které je samo vystaveno kontaktu s fyzickým světem a jinými pevný disk, který je bezpečně skrytý v kovovém těle. Stejně tak se objeví jiné chyby u sekvenčně zapisovaných médií a u médií s náhodným přístupem.

Obecně lze tvrdit, že jsou chyby detekovatené a nedetekovatelné. Detekovatelnost je otázkou použitého detekčního mechanismu, který má určitou sílu a pokud ji chyba převýší, stane se nedetekovatelnou. Snahou tvůrců souborových systémů je pokrytí co největšího množství chyb alespoň jejich detekcí aby bylo možné chráněná data prohlásit za neplatná.

V doplňku k tomu stojí opravitelnost chyb. Stejně jako v předchozím případě opravitelnost stojí na síle použitého algoritmu. A i opravitelná chyba se může stát neopravitelnou, pokud jejich množství překoná schopnosti opravného algoritmu. To, které chyby opravit lze či ne ovšem není možné obecně říct bez znalosti použitých opravných algoritmů. Proto je potřeba pečlivě nadefinovat jaké chyby mohou nastat na daném souborovém systému a mohou být detekovány a kolik z nich je ještě možné opravit pomocí nástrojů poskytnutých tvůrci souborového systému.
 
Ve druhé části práce je popsáno vytvoření nástroje pro vybraný souborový systém \textit{Universal Disk Format}. Je popsána problematika jeho návrhu a následné implementace detekce a opravy jednotlivých definovaných poruch média. K tomuto patří i popsání způsobu otestování výsledného řešení na vhodně zvolených příkladech.

Závěr práce je věnován začlenění vziknuvšího nástroje do komunity GNU a dopad tohoto kroku na \mbox{GNU/Linux}.  
