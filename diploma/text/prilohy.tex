\chapter{Elektronická komunikace s vývojáři UDF}
\section{František Kluknavský} \label{kluknav}
\section{Pali Rohár} \label{pali}
\section{Scott Long \texttt{scott4long@yahoo.com}} \label{scott}
\begin{verbatim}
Hello Scott,

Thanks a lot for reply. I tried to find something from NetBSD project,
but without any luck. It seems they ot stuck at similar point as
FreeBSD. Their status is "not yet finished, but will be added later",
but it was in year 2008 and since then nothing new. I am going to write
to them if they have something...

I am quite wondering what happened between years 2004-2008, because it
seems like UDF died at unix nearly from day to day, distribution to
distribution.

I actually found some open source implementation at OpenSolaris, but it
is only available implementation since OSX's implementation is closed
source same as Microsoft's.

Have a nice day,
Vojtech

On 11.10.2016 01:18, Scott Long wrote:
> Hello Vojtech,
>
> It's nice to hear from you.  Unfortunately I have not worked on UDF in more than 10 years, and I never made much progress on a fsck implementation for it.  You might look at the NetBSD project, I think that they carried the work further forward.  I know that they had talked about fsck_udf at one point, but I'm not sure if they finished it.
>
> Scott
>
>> On Friday, October 7, 2016 8:54 AM, Vojtěch Vladyka <xvlady00@stud.feec.vutbr.cz> wrote:
>> Hello.
>> My name is Vojtech Vladyka and I am master degree student at Faculty of
>> Electrical Engineering and Communications at Brno University of
>> Technology. I am working on fsck for UDF as part of my thesis.
>> I found your homepage https://people.freebsd.org/~scottl/ and there I
>> found you were working on UDF for BSD. I see fsck was between TODOs, but
>> I am wondering if some work weren't done and you just didn't made a 
>> patch.
>> If you have some resources about fsck for UDF or pieces of sources,
>> please let me know.
>> Thanks a lot,
>> Vojtech Vladyka
\end{verbatim}

\section{Reinoud Zandijk \texttt{reinoud@NetBSD.org}} \label{reinoud}
\begin{verbatim}
Hi Vojtech,

On Wed, Oct 12, 2016 at 02:56:02PM +0200, Vojt?ch Vladyka wrote:
> my name is Vojtech Vladyka and I am student at Brno University of
> Technology, Faculty of Electrical Engineering and Communications. I am
> working on my master degree thesis which is making fsck for udf for Linux.
>
> I found old NetBSD thread
> (https://mail-index.netbsd.org/netbsd-announce/2008/05/15/msg000028.html)
> where you are announcing some progress at udf project with note that
> fsck_udf will be added later. At this moment I am gathering implementations
> from other unixes and I haven't found any more info about fsck_udf here. I
> am wondering if you have some unfinished implementation or something you can
> provide to me.

I've created two UDF implementations: UDFclient (www.13thmonkey.org/udfclient)
and the NetBSD UDF implementation. UDFclient was a kind of study into UDF and
far too elaborate to be useful for my FSCK implementation, it'll need to be
pruned first.

UDFclient can read and write rewritable media like CD-(M)RW, DVD*(M)RW, BD-RE
and flash/harddisc up to UDF 2.50 where as the NetBSD UDF implementation can
read and write recordable media too iupto UDF 2.50 and can be considered
superior over UDFclient.

As for fsck_udf, there is only one opensource version and thats the
OpenSolaris one. Before you get too exited, its fairly limited and will only
check older media and even then only deals with recovering free space and get
the directory tree in-order. Important but limited.

I've myself started an fsck_udf but its still in its embryonic stage but I
still want to continue with it.

> If you have some resources about fsck for UDF or pieces of sources, please
> let me know.

I presume you have the full UDF specification documents for UDF? I.e. the
Ecma-167 and all the UDF documents for each version? You can download them at
my stash at http://www.13thmonkey.org/documentation/UDF/. As for SCSI commands
for the media, you'd need the SCSI-MMC documents which you can also download
from my stash at http://www.13thmonkey.org/documentation/SCSI.

Feel free to contact me if you have questions 

With regards,
Reinoud Zandijk
(NetBSD developer)
\end{verbatim}

\chapter{Některé příkazy balíčku \texttt{thesis}}

\section{Příkazy pro sazbu veličin a jednotek}

\begin{table}[!h]
  \caption{Přehled příkazů pro matematické prostředí }
  \begin{center}
  	\small
	  \begin{tabular}{|c|c|c|c|}
	    \hline
	    Příkaz    						& Příklad 					& Zdroj příkladu  							& Význam  \\
	    \hline\hline
	    \verb|\textind{...}|	& $\beta_\textind{max}$ 	& \verb|$\beta_\textind{max}$|	& textový index \\
	    \hline
	    \verb|\konst{...}| 		& $\konst{U}_\textind{in}$ 				& \verb|$\konst{U}_\textind{in}$|		& konstantní veličina \\
	    \hline
	    \verb|\prom{...}| 		& $\prom{u}_\textind{in}$ & \verb|$\prom{u}_\textind{in}$| & proměnná veličina \\
	    \hline
	    \verb|\komplex{...}| 	& $\komplex{u}_\textind{in}$ & \verb|$\komplex{u}_\textind{in}$| & komplexní veličina \\
	    \hline
	    \verb|\vekt{...}| 		& $\vekt{y}$ 						& \verb|$\vekt{y}$| & vektor \\
	    \hline
	    \verb|\matice{...}| 	& $\matice{Z}$ 						& \verb|$\matice{Z}$| & matice \\
	    \hline
	    \verb|\jedn{...}| 		& $\jedn{kV}$ 						& \verb|$\jedn{kV}$|\quad či\ \, \verb|\jedn{kV}| & jednotka \\
	    \hline
	  \end{tabular}
  \end{center}
\end{table}



%\newpage
\section{Příkazy pro sazbu symbolů}

\begin{itemize}
  \item
    \verb|\E|, \verb|\eul| -- sazba Eulerova čísla: $\eul$,
  \item
    \verb|\J|, \verb|\jmag|, \verb|\I|, \verb|\imag| -- sazba imaginární jednotky: $\jmag$, $\imag$,
  \item
    \verb|\dif| -- sazba diferenciálu: $\dif$,
  \item
    \verb|\sinc| -- sazba funkce: $\sinc$.
  \item
    \verb|\mikro| -- sazba symbolu mikro stojatým písmem\footnote{znak pochází z~balíčku \texttt{textcomp}}: $\mikro$.

\end{itemize}
%
Všechny symboly jsou určeny pro matematický mód, vyjma \verb|\mikro|, jenž je použitelný rovněž v~textovém módu.






\chapter{Druhá příloha}

\begin{figure}[!h]
  \begin{center}
    \includegraphics[scale=0.5]{obrazky/ZlepseneWilsonovoZrcadloNPN}
  \end{center}
  \caption{Zlepšené Wilsonovo proudové zrcadlo.}
\end{figure}

Pro sazbu vektorových obrázků přímo v~\LaTeX{}u je možné doporučit balíček \href{https://www.ctan.org/pkg/pgf}{\texttt{TikZ}}.
Příklady sazby je možné najít na \href{http://www.texample.net/tikz/examples/}{\TeX{}ample}.
Pro vyzkoušení je možné použít programy QTikz nebo TikzEdt.




\chapter{Příklad sazby zdrojových kódů}

\section{Balíček \texttt{listings}}

Pro vysázení zdrojových souborů je možné použít balíček \href{https://www.ctan.org/pkg/listings}{\texttt{listings}}.
Balíček zavádí nové prostředí \texttt{lstlisting} pro sazbu zdrojových kódů, jako například:
%
\begin{lstlisting}[language={[LaTeX]TeX}]
\section{Balíček lstlistings}
Pro vysázení zdrojových souborů je možné použít
	balíček \href{https://www.ctan.org/pkg/listings}%
	{\texttt{listings}}.
Balíček zavádí nové prostředí \texttt{lstlisting} pro
	sazbu zdrojových kódů.
\end{lstlisting}
%
Podporuje množství programovacích jazyků.
Kód k~vysázení může být načítán přímo ze zdrojových souborů.
Umožňuje vkládat čísla řádků nebo vypisovat jen vybrané úseky kódu.
Např.:

\noindent
Zkratky jsou sázeny v~prostředí \texttt{seznamzkratek}:
\lstinputlisting[language={[LaTeX]TeX},nolol,numbers=left,firstline=1,lastline=1]{text/zkratky.tex}
%
Šířka textu druhého parametru udává šířku prvního sloupce se zkratkami.
Proto by měla být zadávána nejdelší zkratka nebo symbol.
Příklad definice zkratky \zk{symfvz} je na výpisu \ref{lst:symfvz}.

\iflanguage{czech}{\shorthandoff{-}}{}
\iflanguage{slovak}{\shorthandoff{-}}{}
\lstinputlisting[language={[LaTeX]TeX},frame=single,caption={Ukázka sazby zkratek},label=lst:symfvz,numbers=left,linerange={bsymfvz-\%\%\%\ esymfvz},includerangemarker=false]{text/zkratky.tex}
\iflanguage{slovak}{\shorthandon{-}}{}
\iflanguage{czech}{\shorthandon{-}}{}

\noindent
Ukončení seznamu je provedeno ukončením prostředí:
\lstinputlisting[language={[LaTeX]TeX},nolol,numbers=left,firstnumber=18,linerange=18]{text/zkratky.tex}

\vspace{\fill}

\noindent
{\bf Poznámka k~výpisům s~použitím volby jazyka \verb|czech| nebo \verb|slovak|:}\newline
Pokud Váš zdrojový kód obsahuje znak spojovníku \verb|-|, pak překlad může skončit chybou.
Ta je způsobená tím, že znak \verb|-| je v~českém nebo slovenském nastavení balíčku \verb|babel| tzv.\ aktivním znakem.
Přepněte znak \verb|-| na neaktivní příkazem \verb|\shorthandoff{-}| těsně před výpisem a hned za ním jej vraťte na aktivní příkazem \verb|\shorthandon{-}|.
Podobně jako to je ukázáno ve zdrojovém kódu šablony.


\clearpage

%\section{Výpis kódu prostředí Matlab}
Na výpisu \ref{lst:priklad.vypis.kodu.Matlab} naleznete příklad kódu pro Matlab, na výpisu \ref{lst:priklad.vypis.kodu.C} zase pro jazyk~C.

\lstnewenvironment{matlab}[1][]{%
\iflanguage{czech}{\shorthandoff{-}}{}%
\iflanguage{slovak}{\shorthandoff{-}}{}%
\lstset{language=Matlab,numbers=left,#1}%
}{%
\iflanguage{slovak}{\shorthandon{-}}{}%
\iflanguage{czech}{\shorthandon{-}}{}%
}

\begin{matlab}[frame=single,float=htbp,caption={Příklad Schur-Cohnova testu stability v~prostředí Matlab.},label=lst:priklad.vypis.kodu.Matlab,numberstyle=\scriptsize, numbersep=7pt]
%% Priklad testovani stability filtru

% koeficienty polynomu ve jmenovateli
a = [ 5, 11.2, 5.44, -0.384, -2.3552, -1.2288];
disp( 'Polynom:'); disp(poly2str( a, 'z'))

disp('Kontrola pomoci korenu polynomu:');
zx = roots( a);
if( all( abs( zx) < 1))
    disp('System je stabilni')
else
    disp('System je nestabilni nebo na mezi stability');
end

disp(' '); disp('Kontrola pomoci Schur-Cohn:');
ma = zeros( length(a)-1,length(a));
ma(1,:) = a/a(1);
for( k = 1:length(a)-2)
    aa = ma(k,1:end-k+1);
    bb = fliplr( aa);
    ma(k+1,1:end-k+1) = (aa-aa(end)*bb)/(1-aa(end)^2);
end

if( all( abs( diag( ma.'))))
    disp('System je stabilni')
else
    disp('System je nestabilni nebo na mezi stability');
end
\end{matlab}

\noindent
\begin{minipage}{\linewidth}


%\section{Výpis kódu jazyka C}

\begin{lstlisting}[frame=single,numbers=right,caption={Příklad implementace první kanonické formy v~jazyce C.},label=lst:priklad.vypis.kodu.C,basicstyle=\ttfamily\small, keywordstyle=\color{black}\bfseries\underbar,]
// první kanonická forma
short fxdf2t( short coef[][5], short sample)
{
	static int v1[SECTIONS] = {0,0},v2[SECTIONS] = {0,0};
	int x, y, accu;
	short k;

	x = sample;
	for( k = 0; k < SECTIONS; k++){
		accu = v1[k] >> 1;
		y = _sadd( accu, _smpy( coef[k][0], x));
		y = _sshl(y, 1) >> 16;

		accu = v2[k] >> 1;
		accu = _sadd( accu, _smpy( coef[k][1], x));
		accu = _sadd( accu, _smpy( coef[k][2], y));
		v1[k] = _sshl( accu, 1);

		accu = _smpy( coef[k][3], x);
		accu = _sadd( accu, _smpy( coef[k][4], y));
		v2[k] = _sshl( accu, 1);

		x = y;
	}
	return( y);
}
\end{lstlisting}
\end{minipage}







\chapter{Obsah přiloženého CD}
Nezapomeňte uvést, co čtenář najde na přiloženém médiu.
Je vhodné okomentovat obsah každého adresáře, specifikovat, který soubor obsahuje důležitá nastavení, který soubor je určen ke spuštění atd.
Také je dobře napsat, v~jaké verzi software byl kód testován (např.\ Matlab 2010b).

Pokud je souborů hodně a jsou organizovány ve více složkách,  je možné pro výpis adresářové struktury použít balíček \href{https://www.ctan.org/pkg/dirtree}{\texttt{dirtree}}.

\dirtree{%.
.1 /\DTcomment{kořenový adresář přiloženého CD}.
.2 loga\DTcomment{loga školy a fakulty}.
.3 FEKT-spec-color.eps.
.3 FEKT-spec-color.pdf.
.3 logolink-op\_vavpi.png.
.3 RE-spec-color.eps.
.3 RE-spec-color.pdf.
.3 SIX\_logo\_zahlavi.png.
.2 obrazky\DTcomment{ostatní obrázky}.
.3 soucastky.eps.
.3 soucastky.png.
.3 spoje.eps.
.3 spoje.png.
.3 ZlepseneWilsonovoZrcadloNPN.eps.
.3 ZlepseneWilsonovoZrcadloNPN.png.
.3 ZlepseneWilsonovoZrcadloPNP.eps.
.3 ZlepseneWilsonovoZrcadloPNP.png.
.2 pdf\DTcomment{pdf stránky generované informačním systémem}.
.3 student-desky.pdf.
.3 student-titulka.pdf.
.3 student-zadani.pdf.
.2 text\DTcomment{zdrojové textové soubory}.
.3 literatura.tex.
.3 prilohy.tex.
.3 reseni.tex.
.3 uvod.tex.
.3 vysledky.tex.
.3 zaver.tex.
.3 zkratky.tex.
.2 navod-sablona\_FEKT.pdf\DTcomment{návod na používání šablony}.
.2 obhajoba.tex\DTcomment{hlavní soubor pro sazbu prezentace k~obhajobě}.
.2 readme.txt\DTcomment{soubor s~popisem obsahu CD}.
.2 sablona.tex\DTcomment{hlavní soubor pro sazbu kvalifikační práce}.
.2 thesis.sty\DTcomment{balíček pro sazbu kvalifikačních prací}.
}
