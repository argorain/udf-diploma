\chapter{Ukázkový výpis programu \texttt{udffsck}}
\begin{lstlisting}[frame=single,caption={Ukázka výstupu programu pro médium s veliostí sektoru 512~B},label=lst:bs512,basicstyle=\ttfamily\scriptsize, keywordstyle=\color{black}\bfseries\underbar,nolol,numbers=left]
Device block size: 512
non-option ARGV-elements: ../../udf-samples/bs512-r0150_2.img 
File to analyze: ../../udf-samples/bs512-r0150_2.img
FD: 0x3
[DBG] try #0
[DBG] vsd: type:0, id:BEA01, v:1
[DBG] try #1
[DBG] vsd: type:0, id:NSR02, v:1
[DBG] try #2
[DBG] vsd: type:0, id:TEA01, v:1
bea: type:0, id:BEA01, v:1
nsr: type:0, id:NSR02, v:1
tea: type:0, id:TEA01, v:1
DevSize: 5120000
Current position: 20000
Calc CRC: 0x0878, TagCRC: 0x0878
AVDP[0] successfully loaded.
DevSize: 5120000
Current position: 4e1e00
Calc CRC: 0x0878, TagCRC: 0x0878
AVDP[1] successfully loaded.
DevSize: 5120000
Current position: 4c1e00
Calc CRC: 0x0878, TagCRC: 0x0878
AVDP[2] successfully loaded.

Trying to load VDS
Current position: 20200
Tag ID: 1
VolNum: 1
pVolNum: 0
seqNum: 1
predLoc: 0
New positon is 0x20400
Tag ID: 6
LVD size: 0x1b8
NumOfPartitionMaps: 1
MapTableLength: 6
[0x01] [0x06] [0x01] [0x00] [0x00] [0x00] 
New positon is 0x20600
Tag ID: 5
New positon is 0x20800
Tag ID: 7
VolDescNum: 6
NumAllocDesc: 2
New positon is 0x20a00
Tag ID: 4
New positon is 0x20c00
Tag ID: 8
New positon is 0x20e00
Tag ID: 0
Current position: 4dfe00
Tag ID: 1
VolNum: 1
pVolNum: 0
seqNum: 1
predLoc: 0
New positon is 0x4e0000
Tag ID: 6
LVD size: 0x1b8
NumOfPartitionMaps: 1
MapTableLength: 6
[0x01] [0x06] [0x01] [0x00] [0x00] [0x00] 
New positon is 0x4e0200
Tag ID: 5
New positon is 0x4e0400
Tag ID: 7
VolDescNum: 6
NumAllocDesc: 2
New positon is 0x4e0600
Tag ID: 4
New positon is 0x4e0800
Tag ID: 8
New positon is 0x4e0a00
Tag ID: 0
LVID: loc: 273, len: 512
LVID: lenOfImpUse: 46
LVID: freeSpaceTable: 4613
LVID: sizeTable: 4613
Calc CRC: 0x3501, TagCRC: 0x3501
Calc CRC: 0x5f00, TagCRC: 0x5f00
Calc CRC: 0xaa1b, TagCRC: 0xaa1b
Calc CRC: 0xe0ee, TagCRC: 0xe0ee
Calc CRC: 0x2a01, TagCRC: 0x2a01
Calc CRC: 0x0000, TagCRC: 0x0000
Calc CRC: 0x3501, TagCRC: 0x3501
Calc CRC: 0x5f00, TagCRC: 0x5f00
Calc CRC: 0xaa1b, TagCRC: 0xaa1b
Calc CRC: 0xe0ee, TagCRC: 0xe0ee
Calc CRC: 0x2a01, TagCRC: 0x2a01
Calc CRC: 0x0000, TagCRC: 0x0000
UDF Metadata Overview
=====================
UDF revision: 0
Disc blocksize: 0
Disc blocksize bits: 0
Flags: 0

AVDP
----
[0]
	Identification Tag
	==================
	ID: 2 (AVDP)
	Version: 2
	Checksum: 0x77
	Serial Number: 0x1
	Descriptor CRC: 0x878, Length: 496
	Tag Location: 0x100
[1]
	Identification Tag
	==================
	ID: 2 (AVDP)
	Version: 2
	Checksum: 0xac
	Serial Number: 0x1
	Descriptor CRC: 0x878, Length: 496
	Tag Location: 0x270f
[2]
	Identification Tag
	==================
	ID: 2 (AVDP)
	Version: 2
	Checksum: 0xab
	Serial Number: 0x1
	Descriptor CRC: 0x878, Length: 496
	Tag Location: 0x260f
PVD
---
[0]
	Identification Tag
	==================
	ID: 1 (PVD)
	Version: 2
	Checksum: 0x2d
	Serial Number: 0x1
	Descriptor CRC: 0x3501, Length: 496
	Tag Location: 0x101
[1]
	Identification Tag
	==================
	ID: 1 (PVD)
	Version: 2
	Checksum: 0x50
	Serial Number: 0x1
	Descriptor CRC: 0x3501, Length: 496
	Tag Location: 0x26ff
LVD
---
[0]
	Identification Tag
	==================
	ID: 6 (LVD)
	Version: 2
	Checksum: 0x1a
	Serial Number: 0x1
	Descriptor CRC: 0x5f00, Length: 430
	Tag Location: 0x102
Partition Maps: 1
[1]
	Identification Tag
	==================
	ID: 6 (LVD)
	Version: 2
	Checksum: 0x3e
	Serial Number: 0x1
	Descriptor CRC: 0x5f00, Length: 430
	Tag Location: 0x2700
Partition Maps: 1
PD
--
[0]
	Identification Tag
	==================
	ID: 5 (PD)
	Version: 2
	Checksum: 0xc2
	Serial Number: 0x1
	Descriptor CRC: 0xaa1b, Length: 496
	Tag Location: 0x103
[1]
	Identification Tag
	==================
	ID: 5 (PD)
	Version: 2
	Checksum: 0xe6
	Serial Number: 0x1
	Descriptor CRC: 0xaa1b, Length: 496
	Tag Location: 0x2701
USD
---
[0]
	Identification Tag
	==================
	ID: 7 (USD)
	Version: 2
	Checksum: 0xf5
	Serial Number: 0x1
	Descriptor CRC: 0xe0ee, Length: 24
	Tag Location: 0x104
[1]
	Identification Tag
	==================
	ID: 7 (USD)
	Version: 2
	Checksum: 0x19
	Serial Number: 0x1
	Descriptor CRC: 0xe0ee, Length: 24
	Tag Location: 0x2702
IUVD
----
[0]
	Identification Tag
	==================
	ID: 4 (IUVD)
	Version: 2
	Checksum: 0x29
	Serial Number: 0x1
	Descriptor CRC: 0x2a01, Length: 496
	Tag Location: 0x105
[1]
	Identification Tag
	==================
	ID: 4 (IUVD)
	Version: 2
	Checksum: 0x4d
	Serial Number: 0x1
	Descriptor CRC: 0x2a01, Length: 496
	Tag Location: 0x2703
TD
--
[0]
	Identification Tag
	==================
	ID: 8 (TD)
	Version: 2
	Checksum: 0x3
	Serial Number: 0x1
	Descriptor CRC: 0x0, Length: 496
	Tag Location: 0x106
[1]
	Identification Tag
	==================
	ID: 8 (TD)
	Version: 2
	Checksum: 0x27
	Serial Number: 0x1
	Descriptor CRC: 0x0, Length: 496
	Tag Location: 0x2704
LAP: length: 200, LBN: 3, PRN: 0
LAP: LSN: 274
LogicVolIdent: udf512
FileSetIdent: LinuxUDF
LBNLSN: 274
ROOT LSN: 278
Calc CRC: 0xa05f, TagCRC: 0xa05f

FE, LSN: 278, EntityID: *Linux UDFFS fileLinkCount: 3, LB recorded: 0
LEA 0, LAD 236
01 01 02 00 9c 00 01 00 
dd 9e 18 00 04 00 00 00 
01 00 0a 00 00 02 00 00 
04 00 00 00 00 00 00 00 
00 00 00 00 00 00 00 00 
01 01 02 00 e8 00 01 00 
df e0 20 00 04 00 00 00 
01 00 00 0a 00 02 00 00 
05 00 00 00 00 00 00 00 
10 00 00 00 00 00 08 69 
6d 61 67 65 2e 69 73 6f 
01 01 02 00 e5 00 01 00 
86 36 20 00 04 00 00 00 
01 00 00 08 00 02 00 00 
3a 12 00 00 00 00 00 00 
11 00 00 00 00 00 08 68 
69 73 74 6f 72 79 00 00 
01 01 02 00 21 00 01 00 
06 f2 20 00 04 00 00 00 
01 00 02 07 00 02 00 00 
f4 12 00 00 00 00 00 00 
12 00 00 00 00 00 08 53 
6c 6f 7a 6b 61 00 00 00 
01 01 02 00 0a 00 01 00 
6a 73 24 00 04 00 00 00 
01 00 02 0e 00 02 00 00 
f6 12 00 00 00 00 00 00 
14 00 00 00 00 00 08 50 
72 61 7a 64 6e 61 53 6c 
6f 7a 6b 61 00 00 00 00 

FID found.
Calc CRC: 0x9edd, TagCRC: 0x9edd
FID: ImpUseLen: 0
FID: FilenameLen: 0
ROOT directory
ICB: LSN: 278, length: 512
ROOT ICB: LSN: 278
Actual LSN: 278
Parent. Not Following this one
FLen: 38, padding: 2

FID found.
Calc CRC: 0xe0df, TagCRC: 0xe0df
FID: ImpUseLen: 0
FID: FilenameLen: 10
Filename: image.iso
ICB: LSN: 279, length: 512
ROOT ICB: LSN: 278
Actual LSN: 278
ICB to follow.
Calc CRC: 0x0125, TagCRC: 0x0125

FE, LSN: 279, EntityID: *Linux UDFFS fileLinkCount: 1, LB recorded: 4660
LEA 0, LAD 16
LONG
ExtLen: 4660, ExtLoc: 280
LSN: 4939
00 68 24 00 06 00 00 00 
00 00 00 00 00 00 00 00 

Return from ICB
FLen: 48, padding: 0

FID found.
Calc CRC: 0x3686, TagCRC: 0x3686
FID: ImpUseLen: 0
FID: FilenameLen: 8
Filename: history
ICB: LSN: 4940, length: 512
ROOT ICB: LSN: 278
Actual LSN: 278
ICB to follow.
Calc CRC: 0xd45e, TagCRC: 0xd45e

FE, LSN: 4940, EntityID: *Linux UDFFS fileLinkCount: 1, LB recorded: 185
LEA 0, LAD 16
LONG
ExtLen: 184, ExtLoc: 4941
LSN: 5124
5b 71 01 00 3b 12 00 00 
00 00 00 00 00 00 00 00 

Return from ICB
FLen: 46, padding: 2

FID found.
Calc CRC: 0xf206, TagCRC: 0xf206
FID: ImpUseLen: 0
FID: FilenameLen: 7
Filename: Slozka
ICB: LSN: 5126, length: 512
ROOT ICB: LSN: 278
Actual LSN: 278
ICB to follow.
Calc CRC: 0x84d4, TagCRC: 0x84d4

FE, LSN: 5126, EntityID: *Linux UDFFS fileLinkCount: 2, LB recorded: 0
LEA 0, LAD 148
01 01 02 00 9e 00 01 00 
dd 9e 18 00 f4 12 00 00 
01 00 0a 00 00 02 00 00 
04 00 00 00 00 00 00 00 
00 00 00 00 00 00 00 00 
01 01 02 00 a6 00 01 00 
f7 80 24 00 f4 12 00 00 
01 00 00 0b 00 02 00 00 
f5 12 00 00 00 00 00 00 
13 00 00 00 00 00 08 73 
6f 75 62 6f 72 2e 74 78 
74 00 00 00 01 01 02 00 
fd 00 01 00 48 82 28 00 
f4 12 00 00 01 00 02 11 
00 02 00 00 f7 12 00 00 
00 00 00 00 15 00 00 00 
00 00 08 50 72 61 7a 64 
6e 61 50 6f 64 6c 6f 73 
7a 6b 61 00 00 00 00 00 

FID found.
Calc CRC: 0x9edd, TagCRC: 0x9edd
FID: ImpUseLen: 0
FID: FilenameLen: 0
ROOT directory
ICB: LSN: 278, length: 512
ROOT ICB: LSN: 278
Actual LSN: 5126
Parent. Not Following this one
FLen: 38, padding: 2

FID found.
Calc CRC: 0x80f7, TagCRC: 0x80f7
FID: ImpUseLen: 0
FID: FilenameLen: 11
Filename: soubor.txt
ICB: LSN: 5127, length: 512
ROOT ICB: LSN: 278
Actual LSN: 5126
ICB to follow.
Calc CRC: 0x5e2e, TagCRC: 0x5e2e

FE, LSN: 5127, EntityID: *Linux UDFFS fileLinkCount: 1, LB recorded: 0
LEA 0, LAD 0

Return from ICB
FLen: 49, padding: 3

FID found.
Calc CRC: 0x8248, TagCRC: 0x8248
FID: ImpUseLen: 0
FID: FilenameLen: 17
Filename: PrazdnaPodloszka
ICB: LSN: 5129, length: 512
ROOT ICB: LSN: 278
Actual LSN: 5126
ICB to follow.
Calc CRC: 0x4690, TagCRC: 0x4690

FE, LSN: 5129, EntityID: *Linux UDFFS fileLinkCount: 1, LB recorded: 0
LEA 0, LAD 40
01 01 02 00 cf 00 01 00 
db ce 18 00 f7 12 00 00 
01 00 0a 00 00 02 00 00 
f4 12 00 00 00 00 00 00 
12 00 00 00 00 00 00 00 

FID found.
Calc CRC: 0xcedb, TagCRC: 0xcedb
FID: ImpUseLen: 0
FID: FilenameLen: 0
ROOT directory
ICB: LSN: 5126, length: 512
ROOT ICB: LSN: 278
Actual LSN: 5129
Parent. Not Following this one
FLen: 38, padding: 2

Return from ICB
FLen: 55, padding: 1

Return from ICB
FLen: 45, padding: 3

FID found.
Calc CRC: 0x736a, TagCRC: 0x736a
FID: ImpUseLen: 0
FID: FilenameLen: 14
Filename: PrazdnaSlozka
ICB: LSN: 5128, length: 512
ROOT ICB: LSN: 278
Actual LSN: 278
ICB to follow.
Calc CRC: 0x1a4e, TagCRC: 0x1a4e

FE, LSN: 5128, EntityID: *Linux UDFFS fileLinkCount: 1, LB recorded: 0
LEA 0, LAD 40
01 01 02 00 a0 00 01 00 
dd 9e 18 00 f6 12 00 00 
01 00 0a 00 00 02 00 00 
04 00 00 00 00 00 00 00 
00 00 00 00 00 00 00 00 

FID found.
Calc CRC: 0x9edd, TagCRC: 0x9edd
FID: ImpUseLen: 0
FID: FilenameLen: 0
ROOT directory
ICB: LSN: 278, length: 512
ROOT ICB: LSN: 278
Actual LSN: 5128
Parent. Not Following this one
FLen: 38, padding: 2

Return from ICB
FLen: 52, padding: 0

All done
\end{lstlisting}
\textbf{Legenda} (čísla řádků)
\begin{enumerate}
    \item Zobrazení přijatého parametru velikost sektoru
    \item Ladicí výpis parseru vstupních argumentů
    \item Soubor k analýze
    \item Ladicí informace - File Descriptor v operačním systému
    \item - 10. Ladicí informace ze čtení VRS \addtocounter{enumi}{5}
    \item - 13. Nalezená VRS sekvence \addtocounter{enumi}{2}
    \item - 26. Načtení a kontrola třech AVDP \addtocounter{enumi}{12}
    \item - 75. Načtení hlavní a rezervní VDS a výpis ladicích údajů u některých deskriptorů \addtocounter{enumi}{48}
    \item - 79. Načtení LVID \addtocounter{enumi}{3}
    \item - 91. Ladicí výpis výpočtů a očekávaných hodnot CRC pro VDS \addtocounter{enumi}{11}
    \item - 249. Výpis všech tagů VDS \addtocounter{enumi}{157}
    \item - 257. Načtení a kontrola FSD \addtocounter{enumi}{7}
    \item - 290. Načtení FE, v tomto případě se jedná o kořenový adresář. Výpis bytového pole je položka \texttt{AllocationDescriptors} a ta obsahuje FID jeho potomků. \addtocounter{enumi}{32}
    \item - 301. Výpis FID. Povšimněte si řádku 299, kde je zobrazen ladicí výpis \textit{Parent. Not followng this one.}. Proto je tento FID přeskočen a pokračuje se k dalšímu. Případ s následováním začíná na řádku 302. \addtocounter{enumi}{10}
    \item - 312. Načtení FID. Zde již na řádku 310 je vidět hláška \textit{ICB to follow}, což znamená, že se zanoříme na adresu určenou tímto deskriptorem. \addtocounter{enumi}{10}
    \item - 320. Výpis FE na který ukazoval předchozí FID. Jedná se o soubor, proto jeho \texttt{AllocationDescriptors} obsahují pouze \texttt{long address} dat a ne FID. \addtocounter{enumi}{7}
    \item Ladicí výpis informujicí o úspěšném návratu zpět do rodičovské FE.
    \item Ladicí výpis zobrazující jak dlouhý FID je a o kolik se tím pádem má posunout pro další.
\end{enumerate}
Takto program postupně zpracuje všechny soubory až skončí s hláškou \textit{All done}. Pokud vše skončilo správně, měl by vrátit 0.
\iffalse
\chapter{Příklad sazby zdrojových kódů}

\section{Balíček \texttt{listings}}

Pro vysázení zdrojových souborů je možné použít balíček \href{https://www.ctan.org/pkg/listings}{\texttt{listings}}.
Balíček zavádí nové prostředí \texttt{lstlisting} pro sazbu zdrojových kódů, jako například:
%
\begin{lstlisting}[language={[LaTeX]TeX}]
\section{Balíček lstlistings}
Pro vysázení zdrojových souborů je možné použít
	balíček \href{https://www.ctan.org/pkg/listings}%
	{\texttt{listings}}.
Balíček zavádí nové prostředí \texttt{lstlisting} pro
	sazbu zdrojových kódů.
\end{lstlisting}
%
Podporuje množství programovacích jazyků.
Kód k~vysázení může být načítán přímo ze zdrojových souborů.
Umožňuje vkládat čísla řádků nebo vypisovat jen vybrané úseky kódu.
Např.:

\noindent
Zkratky jsou sázeny v~prostředí \texttt{seznamzkratek}:
\lstinputlisting[language={[LaTeX]TeX},nolol,numbers=left,firstline=1,lastline=1]{text/zkratky.tex}
%
Šířka textu druhého parametru udává šířku prvního sloupce se zkratkami.
Proto by měla být zadávána nejdelší zkratka nebo symbol.
Příklad definice zkratky \zk{symfvz} je na výpisu \ref{lst:symfvz}.

\iflanguage{czech}{\shorthandoff{-}}{}
\iflanguage{slovak}{\shorthandoff{-}}{}
\lstinputlisting[language={[LaTeX]TeX},frame=single,caption={Ukázka sazby zkratek},label=lst:symfvz,numbers=left,linerange={bsymfvz-\%\%\%\ esymfvz},includerangemarker=false]{text/zkratky.tex}
\iflanguage{slovak}{\shorthandon{-}}{}
\iflanguage{czech}{\shorthandon{-}}{}

\noindent
Ukončení seznamu je provedeno ukončením prostředí:
\lstinputlisting[language={[LaTeX]TeX},nolol,numbers=left,firstnumber=18,linerange=18]{text/zkratky.tex}

\vspace{\fill}

\noindent
{\bf Poznámka k~výpisům s~použitím volby jazyka \verb|czech| nebo \verb|slovak|:}\newline
Pokud Váš zdrojový kód obsahuje znak spojovníku \verb|-|, pak překlad může skončit chybou.
Ta je způsobená tím, že znak \verb|-| je v~českém nebo slovenském nastavení balíčku \verb|babel| tzv.\ aktivním znakem.
Přepněte znak \verb|-| na neaktivní příkazem \verb|\shorthandoff{-}| těsně před výpisem a hned za ním jej vraťte na aktivní příkazem \verb|\shorthandon{-}|.
Podobně jako to je ukázáno ve zdrojovém kódu šablony.


\clearpage

%\section{Výpis kódu prostředí Matlab}
Na výpisu \ref{lst:priklad.vypis.kodu.Matlab} naleznete příklad kódu pro Matlab, na výpisu \ref{lst:priklad.vypis.kodu.C} zase pro jazyk~C.

\lstnewenvironment{matlab}[1][]{%
\iflanguage{czech}{\shorthandoff{-}}{}%
\iflanguage{slovak}{\shorthandoff{-}}{}%
\lstset{language=Matlab,numbers=left,#1}%
}{%
\iflanguage{slovak}{\shorthandon{-}}{}%
\iflanguage{czech}{\shorthandon{-}}{}%
}

\begin{matlab}[frame=single,float=htbp,caption={Příklad Schur-Cohnova testu stability v~prostředí Matlab.},label=lst:priklad.vypis.kodu.Matlab,numberstyle=\scriptsize, numbersep=7pt]
%% Priklad testovani stability filtru

% koeficienty polynomu ve jmenovateli
a = [ 5, 11.2, 5.44, -0.384, -2.3552, -1.2288];
disp( 'Polynom:'); disp(poly2str( a, 'z'))

disp('Kontrola pomoci korenu polynomu:');
zx = roots( a);
if( all( abs( zx) < 1))
    disp('System je stabilni')
else
    disp('System je nestabilni nebo na mezi stability');
end

disp(' '); disp('Kontrola pomoci Schur-Cohn:');
ma = zeros( length(a)-1,length(a));
ma(1,:) = a/a(1);
for( k = 1:length(a)-2)
    aa = ma(k,1:end-k+1);
    bb = fliplr( aa);
    ma(k+1,1:end-k+1) = (aa-aa(end)*bb)/(1-aa(end)^2);
end

if( all( abs( diag( ma.'))))
    disp('System je stabilni')
else
    disp('System je nestabilni nebo na mezi stability');
end
\end{matlab}

\noindent
\begin{minipage}{\linewidth}


%\section{Výpis kódu jazyka C}

\begin{lstlisting}[frame=single,numbers=right,caption={Příklad implementace první kanonické formy v~jazyce C.},label=lst:priklad.vypis.kodu.C,basicstyle=\ttfamily\small, keywordstyle=\color{black}\bfseries\underbar,]
// první kanonická forma
short fxdf2t( short coef[][5], short sample)
{
	static int v1[SECTIONS] = {0,0},v2[SECTIONS] = {0,0};
	int x, y, accu;
	short k;

	x = sample;
	for( k = 0; k < SECTIONS; k++){
		accu = v1[k] >> 1;
		y = _sadd( accu, _smpy( coef[k][0], x));
		y = _sshl(y, 1) >> 16;

		accu = v2[k] >> 1;
		accu = _sadd( accu, _smpy( coef[k][1], x));
		accu = _sadd( accu, _smpy( coef[k][2], y));
		v1[k] = _sshl( accu, 1);

		accu = _smpy( coef[k][3], x);
		accu = _sadd( accu, _smpy( coef[k][4], y));
		v2[k] = _sshl( accu, 1);

		x = y;
	}
	return( y);
}
\end{lstlisting}
\end{minipage}

\fi

\chapter{Obsah přiloženého archivu}

Pro úspěšnou kompilaci je potřeba mít nainstalovaný nástroj \texttt{autotools}, \texttt{gcc} verze 4.9 a vyšší a knihovnu \texttt{cmocka} ve verzi 1.1.0. Nástroj je aktuálně testován vůči distribucím Debian a Ubuntu. V tuto chvíli je potřeba pro správnou funkci potřeba little-endian procesor. Tento požadavek bude před dokončením odstraněn.
\dirtree{%.
.1 /\DTcomment{kořenový adresář archivu}.
.2 AUTHORS.
.2 autogen.sh\DTcomment{Skript pro kompilaci balíku}.
.2 cdrwtool\DTcomment{Adresář se zdrojovými kódy nástroje \texttt{cdrwtool}}.
.2 configure.ac.
.2 COPYING.
.2 doc\DTcomment{Adresář s dokumentacemi k jednotlivým nástrojům}.
.2 Doxyfile.
.2 ChangeLog.
.2 include\DTcomment{Adresář se sdílenými hlavičkovými soubory}.
.3 bswap.h.
.3 \detokenize{ecma_167.h}\DTcomment{Soubor se strkuturami podle ECMA-167}.
.3 libudffs.h\DTcomment{Soubor s definicemi funkcí ze sdílené knihovny \texttt{libtool}}.
.3 \detokenize{osta_udf.h}\DTcomment{Soubor se strukturami podle OSTA UDF dokumentace}.
.3 \detokenize{udf_endian.h}.
.3 \detokenize{udf_lib.h}.
.2 INSTALL.
.2 libudffs\DTcomment{Adresář se zdrojovými kódy sdílené knihovy \texttt{libudffs}}.
.2 Makefile.am.
.2 mkudffs\DTcomment{Adresář se zdrojovými kódy nástroje \texttt{mkudffs}}.
.2 NEWS.
.2 pktsetup.
.2 README.
.2 README.md.
.2 TODO.
.2 udffsck\DTcomment{Adresář se zdrojovými kódy nástroje \texttt{udffsck}}.
.3 \detokenize{ecma_119.h}\DTcomment{Soubor s vybranými hlavičkami standardu ECMA-119.}.
.3 main.c\DTcomment{Hlavní soubor programu. Obsahuje kromě funkce main také detekci VRS.}.
.3 Makefile.am.
.3 options.c\DTcomment{Parser vstupních parametrů a nápověda.}.
.3 options.h.
.3 test.c\DTcomment{Zdrojový soubor pro unit-testy.}.
.3 udffsck.c\DTcomment{V tomto souboru je jádro celého programu.}.
.3 udffsck.h.
.3 utils.c\DTcomment{Podpůrné funkce (například tisk tagů)}.
.3 utils.h.
.2 wrudf\DTcomment{Adresář se zdrojovými kódy nástroje \texttt{wrudf}}.
}
