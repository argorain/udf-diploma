\chapter{Souborové systémy OS Linux}
\todo{Srovnani ext3, ext4, udf, fat, ntfs, ... a nastroju pro ně, soustředit se na fsck}
\todo{Vysvětlit co to je souborový systém?}

\section{Fyzická úložná média}
\todo{Ozdrojovat...}
Fyzická úložná média představují nejnižší vrstvu  a zároveň základní kámen všech souborových systémů, protože právě na základě jejich vlastností byly vytvářeny.\\
Z historického hlediska současnost vývoj přešel přes několik technologií počínaje děrnou páskou a děrnými štítky přes magnetickým ukládáním informace do různých nosných médií, optická média po flash paměti. Z hlediska souborových systémů má smysl mluvit až o magnetických médiích, která byla použita pro běžnou práci, protože například lineární pásková úložiště nebyly konstruovány pro kompatibilitu s ostatními systémy ale pouze pro vlastní potřeby, tudíž záznam neobsahoval metadata ale pouze data samotná a jejjich struktura byla uložena jinde nebo vůbec. První opravdové použití souborového systému přišlo až s nástupem disket a pevných disků. \\
První souborový systém v Unixu byl \textbf{S5FS} neboli System V File System (1974), často ovšem označovaný pouze jako \textbf{FS} (File System) nebo \textbf{UFS} (Unix File System). Tento souborový systém byl velice pomalý a také poměrně jednoduchý, ale právě jeho architektura posloužila jako vzor většiny následujících unixových souborových systémů. Dalším významným souborouvým systémem, který vznikl pro diskety byl \textbf{FAT12} (1977). Posléze s rozšířením pevných disků se začaly vyvíjet další nové souborové systémy odpovídající požadavkům.\\
Z hlediska struktury disku lze mluvit o CHS (Cylinder Head Sector), které rozdělilo úložné médium ve třech rozměrech podle jednotlivých stop (Cylinder), čtecí hlavy (Head), kterých může být u pevných disků více, záleží na počtu jednotlivých disků a použitých stran a poslední rozměr je sektor (Sector), který určuje o kterou část stopy jde. Právě sektory dostály ve vývoji největších změn, protože původní dělení po úhlech od středu disku nebylo efektivní se vzrůstající vzdáleností od středu. Proto se později přešlo k dělení na jednotnou velikost sektoru, které se rozložily rovnoměrně po celém disku.\\
Sektor je nejmenší zápisovou jednotkou na médiu, která je určená výrobcem média. Typicky se jedná o velikosti 512~B u pevných disků, 2048~B u CD-ROM a DVD-ROM. Nové pevné disky používají velikost bloku 4096~B. Důvodem pro členění po sektorech je zjednodušení přístupu k médiu za předpokladu, že uložená data jsou větší než velikost bloku. V opačném případě dochází k plýtvání místem, protože se vždy čte a zapisuje celý blok. Právě proti sektorům se obvykle navrhují bloky souborového systému, které se snaží využít sektory co nejefektivněji jak z hlediska úspory místa tak z hlediska přístupového a zápisového času.\\

\section{Diskové oddíly}
Diskové oddíly představují způsob, jak lze fyzické médium abstraktně rozdělit na více částí. Rozdělení na oddíly je uloženo v MBR (Master Boot Record) a jeden disk může obsahovat až 4 hlavní oddíly. V případě potřeby více oddílů lze použít rozšířený oddíl, který zapouzdřuije další dělení na oddíly, ovšem tentokrát již mimo MBR.\\
Důvodem pro rozdělení na oddíly může být oddělení uživatelských dat od systémových, vyhrazení místa pro odkládací oddíl nebo jiné logické dělení. 

\section{Logical Volume Manager (LVM)}
\todo{https://wiki.archlinux.org/index.php/LVM}
Logical Volume Manager (LVM) představuje způsob abstrakce nad diskovými oddíly. Ačkoli se může zdát, že se jedná o podobnou technologii jako je RAID, není tomu tak (\cite{https://wiki.archlinux.org/index.php/Software_RAID_and_LVM}).\\
LVM vzniklo pro zajištění větší flexibility pro organizaci úložiště. Představte si situaci na následujícím obrázku:
\begin{verbatim}
Disk1 (/dev/sda):
 _ _ _ _ _ _ _ _ _ _ _ _ _ _ _ _ _ _ _ _ _ _ _ _ _ _ _ _ _ _ _ _ _ _ _ _ _
|Partition1 50GB (Physical volume) |Partition2 80GB (Physical volume)     |
|/dev/sda1                         |/dev/sda2                             |
|_ _ _ _ _ _ _ _ _ _ _ _ _ _ _ _ _ |_ _ _ _ _ _ _ _ _ _ _ _ _ _ _ _ _ _ _ |
                                                     
Disk2 (/dev/sdb):
 _ _ _ _ _ _ _ _ _ _ _ _ _ _ _ _ _ _ _ _ _ _ _ _ _ _
|Partition1 120GB (Physical volume)                 |
|/dev/sdb1                                          |
| _ _ _ _ _ _ _ _ _ _ _ _ _ _ _ _ _ _ _ _ _ _ __ _ _|
\end{verbatim}
Jak je vidět, situace obsahuje dva disky, každý o jiné velikosti a formátu. Pokud bychom chtěli vytvořit diskový oddíl, museli bychom sáhnout po řešení RAID 0 a disky posléze naformátovat dle potřeby. V případě změny velikosti by bylo nutné uložiště přeformátovávat. Ovšem při použití LVM situace vypadá například takto:
\begin{verbatim}
Volume Group1 (/dev/MyStorage/ = /dev/sda1 + /dev/sda2 + /dev/sdb1):
 _ _ _ _ _ _ _ _ _ _ _ _ _ _ _ _ _ _ _ _ _ _ _ _ _ _ _ _ _ _ _ _ _ _ _ _ _ 
|Logical volume1 15GB  |Logical volume2 35GB    |Logical volume3 200GB    |
|/dev/MyStorage/rootvol|/dev/MyStorage/homevol  |/dev/MyStorage/mediavol  |
|_ _ _ _ _ _ _ _ _ _ _ |_ _ _ _ _ _ _ _ _ _ _ _ |_ _ _ _ _ _ _ _ _ _ _ _ _|
\end{verbatim}
Všechny oddíly byly sloučeny do jedné logické skupny s názvem \texttt{MyStorage} s velikostí 250~GB a to bylo následně pouze logicky rozděleno na oddíly jak je vidět na obrázku. Tyto logické oddíly posléze budou naformátovány libovolným souborovým systémem.\\
Zásadní rozdíl oproti RAID 0 je v možnostech změny struktury úložiště. V případě RAID 0 se změna promítá do formátování RAID úložiště, v případě LVM se jedná o změnu velikosti logické jednotky pouze za předpokladu existence nějakého volného místa, nezávisle na jeho místě na disku. Tudíž není třeba přesouvat oddíly pro potřeby jejich zvětšení.

\section{Srovnání souborových systémů}
Pro operační systém Linux bylo během jeho existence vyvinuto mnoho různých souborových systémů stejně tak jako jich bylo mnoho implementováno z ostatních operačních systémů. Některé z nich jsou používány i v současnosti, některé byly opuštěny nebo nahrazeny.\\
Podle \cite{tldp-filesystem} patří mezi nejvýznamnější nativní souborové systémy tyto:
\begin{itemize}
    \item minix -- nejstarší souborový systém OS Linux. Vzhledem k jeho stáří se dnes již aktivně nevyvýjí a nepoužívá. Mezi jeho hlavní omezení patří maximální velikost souborového systému, která je až 64~MB. Dalším významným omezením je délka názvů soborů, která je limitována na 30 znaků.
    \item ext -- tento souborový systém, celým jménem Extended Filesystem, vznikl v roce 1992 speciálně pro potřeby kernelu OS Linux \cite{wiki-ext}. Jeho inspirací byl souborový systém UFS (Unix File System), ze kterého převzal strukturu metadat. Jeho limitací byla maximlní velikost 2~GB.
    \item ext2 -- nástupce souborového systému ext. Byla navýšena maximální velikost na 32~TB a byl navržen podle stejných principů jako Berkeley Fast File System z projektu BSD \cite{wiki-ext2}. Dlouhou dobu byl tento souborový systém používán ve všech hlavních distribucích OS Linux. ext2 nebyl zpětně kompatibilní s původním ext.
    \item ext3 -- již v pořadí třetí verze ext souborového systému. Nyní již byla zachována téměř plná zpětná kompatibilita s ext2. Oproti ext2 ovšem ext3 přineslo žurnálování, které pomohlo v obnově souborového systému při chybě systému \cite{wiki-ext3}. Zbytek parametrů zůstal shodný s ext2.
    \item ext4 -- současná verze ext souborového systému \cite{wiki-ext4}. Opět je zpětně kompatibilní s ext3 a ext2 ale oproti nim byla navýšena maximální velikost na 1~EB, ale doporučené maximum je 16~TB. Původně byl ext4 pouze skupina rozšížení pro ext3, ale později se osamostatnil jako ext4. Dalším rozšířením je nyní již neomezený počet podsložek (ext3 mělo omezení na 32000 podsložek). Také bylo zlepšena odolnost žurnálu přidáním kontrolních součtů a celková výkonnost souborového systému.
    \item vfat -- typický zástupce cizích souborových systémů. V tomto případě se jedná o MS FAT32, který je hojně používaný na USB flash discích.
    \item iso9660 -- standardní souborový systém pro CD-ROM. 
    \item udf -- souborový systém původně vyvynutý pro datové rozšíření CD-ROM. Posléze rozšířen pro DVD a BluRay. V současnosti je používán například v šifrovaných flash discích. 
    \item smbfs -- síťový souborový systém Samba File Systém vyvynutý pro sdílení dat mezi počítači s MS Windows. Podporuje Windows Sharing Protocol.
    \item ntfs -- v současnosti nejpoužívanější souborový systém v MS Windows.
\end{itemize}

\section{Nástroje kontroly integrity}
V prostředí OS Linux je pro kontrolu integrity určen primárně nástroj \texttt{fsck} (Filesystem Consistency Check). V manuálové stránce k \texttt{fsck} \cite{man-fsck} se lze dočíst, že samotný \texttt{fsck} je pouze frontend, který volá specifické nástroje pro daný souborový systém. Skutečné nástroje určené pro kontrolu integrity, které jsou obvykle dodávány spolu s nástroji k vytvoření a práci s daným soborovým systémem se jmenují \texttt{fsck.fstype}, například \texttt{fsck.ext4}.\\
Pro masivně používané souborové systémy tyto nástroje existují. Pro \textbf{ext} rodinu se jedná o nástroj \texttt{e2fsck}. Podporovány jsou i převzaté souborové systémy, například \texttt{fsck.fat} pro souborové systémy \textbf{vfat} a \textbf{msdos}. Ovšem pro například pro \textbf{iso9660} nástroje kontrolující integritu existují pouze jako nástroje vzniklé na základě pokusů při testování vadných disků. Takovým je i \texttt{isovfy} \cite{man-isovfy} ze skupiny nástrojů isoinfo, který ovšem pouze umožňuje kontrolu integrity, nikoli opravu chyb.\\
Nástrojem který v tuto chvíli chybí je podle všeho \texttt{fsck.udf}. Projekt ve s názvem udftools \cite{udftool-sourceforge} ve kterém vznikl i podprojekt udffsck byl autorem opuštěn v roce 2004 a nástroj pro kontrolu konzistence nebyl nikdy započat. 

\chapter{Universal Disk Filesystem}
\todo{Popsat filesystem, způsob uložení dat, ochrané mechanismy... UDF docs rulez!}
Universal Disk Filesystem (UDF) je souborový systém navržený pro použití na CD, DVD a Blu-Ray discích ale díky jeho univerzálnosti je možné jej použít i na jiných médiích. Jeho návrh vyšel ze systému CDFS (ISO 9660) ale není s ním kompatibilní. Jednou z výhod UDF oproti CDFS je například možnost dodatečného zápisu na CD anebo naopak mazání dat z CD. 

\section{Struktura systému}
%Po vzoru CDFS byla převzata struktura začátku souborového systému pro snadné a rychlé rozlišení CDFS a UDF. Oba totiž sdílejí výchozí velikost sektoru, která je 2048~B a pro boot sektor je vyhrazeno prvních 16 bloků. Poté u obou následují popisovače systému, které jsou zarovnané po 2048~B.\\ \todo{je to tak vzdy? nebo to je zarovnané na velikost sektoru?}
Pro potřeby popisu struktury systému je potřeba zavést tři číslování. Prvním je \textit{Logical Sector Number} (LSN), druhým je \textit{Logical Block Number} (LBN) a třetím je LSN z ISO9660. Ve většině případů bude jejich velikost shodná (a norma popisující UDF to i vyžadauje) ale může se stát že se velikosti neshodují. Naopak v případě LSN převzatým z ISO9660 je velikost sektoru fixně 2~kB. V tom případě se jejich umístění přestane překrývat a je třeba je začít rozlišovat.\\
Na začátku média nebo stopy je vyhrazeno 16 sektorů o velikosti 2~kB (takže celkem 32~kB) pro potřeby operačního systému. Obvykle není tato část využita je vyplněna nulami.\\
Po vzoru CDFS byla vytvořena úvodní část s názvem \textit{UDF Bridge Volume Recognition Sequence} (VRS) začínající na fixním sektoru 16 (počítáno právě podle ISO9660), jejíž částí může být ISO9660 \textit{Primary Volume Descriptor} a \textit{Volume Descriptor Terminator} nebo \textit{Boot Volume Descriptor}. Poté následuje v případě že je přítomno UDF deskriptory \textit{Beginning Extended Area Descriptor}, \textit{NSR Descriptor} a \textit{Terminating Extended Area Descriptor}. Je to právě \textit{NSR Descriptor}, který určuje přítomnost UDF na médiu. Pro VRS je vyhrazena velikost 6 ISO9660 sektorů.\\
Poté následuje rezervované místo pro potřeby ostatních deskriptorů UDF. Toto místo vyplňuje oblast od konce VRS až do druhého fixního sektoru 256, kde je umístěn \textit{Anchor Volume Descriptor Pointer} (AVDP). Fyzicky je tato adresa počítána podle UDF LSN od začátku média (!) a nikoli od předchozích LSN, které jsou o velikosti podle ISO9660. AVDP je výchozím bodem UDF. Právě z něj míří dva ukazatele na sektory obsahující \textit{Main Volume Descriptor Sequence} a \textit{Reserve Volume Descriptor Sequence}. Tyto oblasti mají obě fixní velikost 16 sektorů. Jak je vidět, volné místo mezi koncem VRS a AVDP je větší, zbytek tohoto místa by měl být vyplněn nulami. Samotné umístění deskriptorových sekvencí není striktně předepsáno ale bývá zarovnáno s sektorem 32.\\
\textit{Main (Reserve) Volume Descriptor Sequence} obsahují duplicitní informace o samotném UDF. Jejich deskriptory budou popsány v následujících kapitolách. Již předešlu že uvnitř je uložena velikost logického bloku, který je používaný dále.\\
Poslední oblastí, která je adresovaná podle sektorů je \textit{Path Table / Directory Record}, který začíná na sektoru 257 a končí na sektoru 257~+~65536 jak udává norma. Ovšem v tomto případě jsou známé problémy s dodržením tohoto limitu a může dojít i k překročení této velikosti. Není to situace pokrytá normou ale je potřeba ji mít na paměti. Tato oblast je popsána pomocí standardu ISO9660 (ECMA-119) a obsahuje informace o všech kořenových adresářích.

\subsection{UDF Bridge Recogniton Sequence}
Tato sekce je určena k rozpoznání obsaženého souborového systému. Jak bylo předesláno v předchozí kapitole, může být velká až 6 sektorů a obsahuje deskriptor identifikující souborový systém. Jeho poloha je fixní na sektoru 16.\\
Struktura deskriptorů je následující:
\begin{center}
    \begin{tabular}{ | l | l | l | l | l | }
        \hline
        Adresa  & Délka [B]   & Jméno položky & Datový typ    & Popis \\ \hline
        0       & 1             & Type           & int8          & Typ deskriptoru \\ \hline
        1       & 5             & Identifier & string        & Identifikátor typu souborového systému \\ \hline
        6       & 1             & Version         & int8          & Verze deskriptoru \\ \hline
        7       & 2041          & ---           & ---           & Volné místo, záleží na typu deskriptoru \\ \hline
    \end{tabular}
\end{center}
Jak je vidět, deskriptor se skládá ze třech částí, pomineme-li poslední rezervovanou část obsahující volné místo. První položka \textit{Type} určuje typ deskriptoru. Výčet možných typů je uveden v tabulce \todo{Type Codes table}. Další položka s názvem \textit{Identifier} obsahuje řetězec o pěti znacích. Tato část určuje typ použitého souborového systému. Možné varianty jsou tyto:
\begin{itemize}
    \item \texttt{BOOT0} - Bootovací záznam
    \item \texttt{CD001} - souborový systém ISO~9660
    \item \texttt{CDW02} - TODO
    \item \texttt{BEA01} - \textit{Beginning Extended Area Descriptor}, začátek UDF Bridge sekvence
    \item \texttt{TEA01} - \textit{Terminating Extended Area Descriptor}, konec UDF Bridge sekvence
    \item \texttt{NSR01} - UDF verze 1.0 a vyšší
    \item \texttt{NSR02} - UDF verze 1.5 a vyšší
    \item \texttt{NSR03} - UDF verze 2.0 a vyšší
\end{itemize}
Poslední položkou je \textit{Version} která obsahuje inforamci o verzi deskriptoru. Zde by měla být v těchto případech vždy 0x01.\\
Aby mohlo být prohlášeno, že médium obsahuje UDF, musí být nalezen deskriptor obsahující \textit{Identifier} s hodnotou \texttt{NSR01}, \texttt{NSR02} nebo \texttt{NSR03}. V opačném případě se nejedná o UDF.

\subsection{Anchor Volume Descriptor Pointer}
\textit{Anchor Volume Descriptor Pointer} (AVDP) je klíčová část souborového systému. Tato struktura udržuje adresu deskriptorů souborového systému a jejich délku. Právě AVDP je první deskriptor, který se čte po identifikaci souborového systému a proto je na fixním sektoru číslo 256 a také v případě ukončených souborových systémů se nachází jeho kopie na posledním sektoru. Jeho struktura je vidět na tabulce \ref{tab-avdp}. Jak je vidět, obsahuje ukazatele na hlavní a záložní VDS což zvyšuje robustnost.
\begin{table}
    \begin{tabular}{ | l | l | p{4.5cm} | p{1.3cm} | p{5.5cm} | }
        \hline
        Adresa  & Délka [B]   & Jméno položky & Datový typ    & Popis \\ \hline
        0       & 16          & DescriptorTag & struct        & Popis deskriptoru \\ \hline
        16      & 8           & MainVolumeDescriptor SequenceExtent & struct & Sektor a délka \textit{Main Volume Descriptor Sequence} \\ \hline
        24      & 8           & ReserveVolumeDescriptor SequenceExtent & struct & Sektor a délka \textit{Reserve Volume Descriptor Sequence} \\ \hline
        32      & 480         & Rezerva & & \\ \hline
    \end{tabular}
    \caption{Anchor Volume Descriptor Pointer\label{tab-avdp}}
\end{table}

\subsection{Volume Descriptor Sequence}
\textit{Volume Descriptor Sequence} (VDS) je skupina deskriptorů popisující veškeré vlastnosti souborového systému UDF. Jsou seřazeny postupně na jednotlivých sektorech počínaje adresou udanou z AVDP \ref{s-avdp} a končí po počtu uloženém opět v AVDP. Jejich pořadí není fixní, jsou identifikovány pomocí \textit{DescriptorTag}.S postupným vývojem UDF přibývaly i deskriptory v VDS.\\
Jak je vidět v AVDP, i tato část je uložena duplicitně pro zvýšení robustonosti. Rezervní VDS je uložena ve stejném tvaru jako primární na následucích sektorech.

\subsection{Logical Volume Integrity Sequence}

\subsection{ECMA-119 File Structure}

\subsection{UDF File Structure}

\subsection{File Data Structure}

\chapter{Definice chyb na souborovém systému}
\todo{Jak se to může pokazit? A co se s tím dá dělat?}
Chyby na souborovém systému mohou vzniknout ze třech příčin. 
\begin{enumerate}
    \item Chybou ovladače souborového systému,
    \item nekorektním odpojením souborového systému (například odpojení flash disku před ukončením všech transakcí),
    \item fyzickým poškozením média (například stářím poškozené bloky flash paměti nebo poškrábané optické médium).
\end{enumerate}
První druh chyb se děje zřídka. Důvodem je skutečnost, že programy a kernelové moduly starající se o přístup k a práci se souborovými systémy bývají dobře odladěné a otestované. Koneckonců, právě data jsou to, co má v počíačích hodnotu.\\
Druhý druh chyb se vyskytuje velice často. Odebrání disku ve spěchu bez korektního odpojení může způsobit poškození souborového systému skrz přerušení probíhající zápisové operace. Systém poté zůstane v nekonzistentním stavu, protože se tam nachází částečně zapsaný soubor. Případně může dojít k poškození metadat souboru. Do této kateogie spadají i chyby vzniklé pádem operačního systému.\\
Třetí kategorií chyb jsou veškeré poruchy fyzického média. U optických disků se první vybaví škrábance a rýhy. Tyto chyby poté bývají shluknuty do clusterů poškozených dat. U magnetických disků může dojít k poškození kolizí čtecích hlav s plotnou nebo k poškozením opotřebením. V dnešní době se dá předejít obojímu. Vyšší řady disků určených pro notebooky mají integrovaný akcelerometr a v případě většího zrychlení dojde k nouzovému zaparkování čtecích hlav. Chybám z opotřebení lze předcházet pomocí integrovaného systému S.M.A.R.T. který se stará o sběr telemetrických dat o disku a na jejich základě lze předvídat jeho poruchu. S příchodem FLASH technologie se objevil druh chyb ve formě vadných paměťových buněk, ať už z výroby nebo opotřebením. V jistém malém procentu může dojít i k vytvoření chyby pomocí vysoce nabité částice, která změní napěťový potenciál v bitu.\\
Opravitelnost a analýza chyb je vždy otázkou míry a typu poškození. 

\chapter{Realizace nástroje pro detekci chyb}
\todo{Vlastní řešení.}
\todo{https://github.com/illumos/illumos-gate/tree/master/usr/src/cmd/fs.d/udfs/fsck}
\section{Stav projektu udftools}
Projekt udftools byl založen roce 1999 na projektovém serveru SourceForge \cite{udftools-sourceforge} Bennem Fennemou \ref{fennema}. Byl implementován standard UDF 1.5 a byly vytvořeny nástroje pro paketový zápis na CD, vytvoření souborového systému UDF a pro přístup k němu. Nástroj pro kontrolu konzistence zůstal jako prázdný projekt, byly ovšem vytvořeny nástroje pro vytvoření souborového systému UDF, nástroje pro jeho zápis včetně paketového zápisu.\\
V roce 2007 byly integrovány poslední patche a poté projekt zůstal ležet ladem. V roce 2014 byl projekt přemigrován na GitHub \cite{udftools-github} Palim Rohárem \ref{rohar} a znovu byl započat vývoj, převážně opravy starých chyb. Byla implementována verze UDF 2.01. Nástroj \texttt{fsck.udf} nebyl v tomto projektu nikdy vytvořen a tento stav přetrvává do současnosti.\\
V tuto chvíli se komunita okolo projektu udftools skládá hlavně z Paliho Rohára \ref{rohar} a z Františka Kluknavského \ref{kluknavsky}. Vzhledem k nepříliš vysoké popularitě UDF není projekt udftools aktivní. Sice je stále udržován panem Rohárem, ale aktivní vývoj v tuto chvíli pravděpodobně neprobíhá, nebo alespoň ne veřejně.

\section{Existující nástroje pro kontrolu konzistence UDF}
\todo{Předělat citace do příloh na citace v textu}
Pro OS Linux jsou znám pouze nástroj \texttt{udfct\_1.5r4}. Tento nástroj byl vyvynut firmou Philips a jedná se o nástroj který je schopný zkontrolovat integritu souborového systému UDF. Chyby v implementaci UDF a případné chyby v datech vypisuje do terminálu ale není schopný je opravovat. Zásadním problémem tohoto nástroje je, že existuje pouze pod restriktivní licencí firmy Philips a není jej tudíž možné použít jako výchozí bod dalšího vývoje i přes dostupnost zdrojových souborů. V současnosti je balíček se zdrojovými kódy dostupný na Wayback Machine \cite{wayback}, firma Philips jej již nenabízí. Mou žádost o poskytnutí práva na přepoužití jejich kódu ignorovali.\\
V BSD je \texttt{fsck} v podobném stavu jako v Linuxu ovšem komunita okolo BSD na problému pomalu pracuje. Našel jsem blog \cite{scottuvblog} od Scotta Longa z projektu FreeBSD kde má shrnutí své práce na UDF včetně jeho patchů do FreeBSD implementace UDF. Mezi jeho body k doplnění je i nástroj \texttt{fsck} a právě proto jsem ho oslovil s prosbou o informace. Elektronická komunikace mezi mnou a jím je v příloze \ref{scott}. Právě Scott Long mne odkázal na projekt NetBSD, protože on sám nikdy na \texttt{fsck} nezačal pracovat, ale doslech se, že NetBSD něco má. Po hledání jsem nalezl mailing list z roku \todo{rok mailing listu}, kde Reinoud Zandijk popisuje svůj postup práce na UDF implementaci. Na konci jeho zprávy je poznámka o \texttt{fsck} s informací, že bude "brzy". Žádná implementace se ovšem na svět nedostala, takže jsem mu napsal (viz \ref{reinoud}) s prosbou o informace o  stavu jeho implementace \texttt{fsck}. Jeho odpověď byla vyčerpávající a potvrzovala mé podezření o stavu open source implementace. On sám má rozpracovanou implementaci \texttt{fsck} ale zveřejňovat ji zatím nebude, protože není dokončená. Nikdo další žádnou jinou open source implementaci nemá vyjma projektu Open Solaris.\\
Implementace v projektu Open Solaris je tedy v současnosti jedinou dostupnou open source implementací \texttt{fsck}. Jejich kód je dostupný na serveru GitHub \cite{solaris-github}. Bohužel ani jejich implementace není kompletní a dokáže obnovit pouze volné místo a strom souborového systému ale bez dat. Dalším omezením je maximální verze UDF, která je omezena na \todo{Open Solaris max UDF version}. Ovšem i toto je dobrý výchozí bod pro další práci.\\
Co se týká OSX od společnosti Apple, ti mají nástroje pro kontrolu UDF implementovanou v nástroji \texttt{fsck\_udf} pro všechny exitující revize, ovšem zdrojové kódy nejsou veřejně dostupné, takže je možné jejich nástroje použít pouze jako referenční pro srovnání funkčnosti.\\
Microsoft Windows má implementovaný checkdisk pro UDF ale jejich implementace trpí problémy s implementací samotného ovladače pro UDF, který není schopný načíst souborové systémy větší než 1~GB.\\

 
