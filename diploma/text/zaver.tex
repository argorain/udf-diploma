\chapter{Závěr}
Cílem práce bylo vytvořit nástroje pro diagnostiku integrity dat souborového systému pro OS Linux, pro který v současnosti tato podpora neexistuje.

Byla provedena analýza úložných řešení používaných v GNU/Linux z hlediska jejich struktury, funkce a možností kontroly konzistence. Bylo zjištěno, že nástroje pro kontrolu konzistence jsou zapouzdřeny pod nástrojem \texttt{fskc}, který volá specifické nástroje daného souborového systému. Cílem nástroje \texttt{fsck} je kontrola a oprava metadat (deskriptorů) souborového systému a ne oprava dat samotných.

Během rešerše souborových systémů používaných v OS Linux bylo zjištěno, že neexistuje nástroj \texttt{fsck} pro souborový systém UDF. Proto jsem zvolil tento souborový systém pro další práci. Tomu předcházela analýza tohoto souborového systému z hlediska jeho funkce, použití a přístupu k detekci a korekci chyb.

Na základě výběru souborového systému UDF byly definovány detekovatelné a opravitelné chyby kterými má smysl se zabývat. V zásadě jde o chyby v metadatech souborového systému, protože data samotná jsou chráněna kontrolními a opravnými mechanismy nosného média, typicky pomocí ECC bloků. Metadata jsou na úrovni UDF chráněna kombinací kontroly skutečné a deklarované polohy na médiu, která umožňuje odhalit případné chyby vzniklé při zápisu deskriptoru, kontrolních součtů a CRC, jež zajišťují konzistenci deskriptorů jako takovou. Další implementovaný mechanismus souborového systému UDF je redundance klíčových deskriptorů. Tím je zvýšena šance na zachování nejdůležitějších deskriptorů souborového systému, tedy deskriptorů popisující médium jak celek. Klíčovou vlastností je vyřešení detekce a následné korekce neukončeného zápisu. Neukončený zápis lze detekovat v deskriptoru \textit{Logical Volume Integrity Descriptor} nebo pomocí ostatních indikátorů jako například lišící se velikost souboru vůči jeho deklarované velikosti nebo nenastavné \textit{Unique ID}. Všechny tyto způsoby jsou detailně popsány v kapitole \ref{ch:definice-chyb}. Poslední částí jsou korekce detekovaných chyb v deskriptorech stromu souborů. Tyto chyby úzce souvisí s přerušeným zápisem ale mohou vzniknout i jiným způsobem, například přehřátím buňky flash paměti nebo poškozeným kabelem vedoucím k médiu. Jedná se vedle chyb v deklarované poloze, kontrolním součtu a CRC o chyby v unikátním identifikátoru \textit{Unique ID} každého souboru, respektive jeho deskriptorů \textit{File Entry} a \textit{File Identifier Descriptor}, chyby v sériovém čísle tagů deskriptorů, neaktualizovaný čas posledního zápisu nebo neshodující se deklarovaný a skutečný objem zaznamenaných dat.

Většinu detekovaných chyb je možné opravit. K tomu slouží korekční část nástroje. Mezi opravitelné poruchy patří poškozený \textit{Anchor Volume Descriptor Pointer}, což je vstupní bod do souborového systému. Stejně tak je opravitelná \textit{Volume Descriptor Sequence}, která popisuje médium jak celek. Chyby provozního rázu v \textit{Logical Volume Integrity Descriptor} opravit lze ale pokud je deskriptor poškozen strukturálně (t.j. selže porovnání kontrolního součtu nebo CRC), opravitelný v tuto chvíli není a bylo by nutné jej rekonstruovat. Opravy chyb ve stromu souborů se opět týkají nestrukturálních poruch. Pokud je deskriptor poškozen strkukturálně, je poškozená část souborového stromu odstraněna. Pokud se jedná o provozní poškození (například nekonzistence v sériových číslech tagů, neplatné časové značky nebo nekonzistentní \textit{Unique ID}), je možné tyto chyby bez potíží odstranit.

Práce na nástroji \texttt{udffsck} byla započata v rámci existujícího balíčku \texttt{udftools}, který je součástí všech distribucí GNU/Linux. Je překvapivé, že do této doby nebyl tento nástroj vytvořen, vzhledem k absenci jiného funkčního korekčního řešení pro souborové systémy UDF. Nástroj \texttt{udffsck} jsem vytvořil v jazyce C podle standardu C99 s využitím knihovních funkcí poskytnutých operačním systémem \mbox{GNU/Linux}. Nástroj je přeložitelný s minimálními nároky na hostitelský systém stejně jako zbytek balíčku. 

Mnou navržený nástroj \texttt{udffsck} slučuje jak detekční, tak korekční funkci. To znamená, že dokáže nejen detekovat všechny definované chyby na médiích naformátovaných souborovým systémem UDF až do verze standardu 2.01, ale také provádět jejich opravy. Implementace samotná intenzivně využívá mapování média do paměti a tím sdíleného paměťového prostoru (\textit{Shared Memory}) pro zjednodušení práce s ním. Díky tomu byly eliminovány chyby vzniklé přesouváním ukazetele po médiu a je umožněno přímé mapování deskriptorů na médium.

Mnou vytvořené řešení v podobě nástroje \texttt{udffsck} bylo otestováno vůči výběrové množině testovacích dat o objemu 14~GB a 29 vzorcích, které zachycují nejčastěji se vyskytující verze UDF a jejich nejčastější nastavení. Většina testovacích dat byla vytvořena způsobem popsaným v podkapitole \ref{sec:data}, t.j. vytvářením obrazů disků pomocí různých nástrojů poskytovaných prostředním \mbox{GNU/Linux}, Microsoft Windows a Apple macOS a jejich následným poškozováním a menší část byla převzata z testovací sady nástroje \textit{blkid}. Významná část testovacích dat (přibližně o objemu 10~GB) byla vytvořena experimentálním používáním souborového systému s cílem napodobit životní cyklus média v reálném nasazení. Tento přístup pomohl odhalit řadu potíží, které nebylo možné odhalit syntetickým vytvářením poruch. Na základě těchto dat jsou vytvořeny automatizované testy, které jsou spouštěny po každém nahrání do repozitáře na server GitHub (viz. můj repozitář \cite{udftools-github-moje}) a testují, zda nedošlo k poškození dříve funkčních částí. Tímto způsobem bylo odhalena řada chyb, které by jinak mohly zůstat přehlédnuty. V rámci testování jsem otestoval širokou škálu možností, které souborový systém UDF poskytuje.

Výsledky práce byly publikovány v komunitě GNU v podobě začlenění výsledného nástroje \texttt{udffsck} do existujícího balíčku \texttt{udftools}. Tento proces předpokládal splnění nejen požadavků zadání této práce, ale i určité kvality kódu a jeho testů. Vzhledem k faktu, že od začátku tato práce směřovala k tomuto cíli, byla v rané fázi práce navázána komunikace se správcem repozitáře panem Palim Rohárem a postup práce byl konzultován s ním. Začleněním nástroje \texttt{udffsck} do balíčku \texttt{udftools} je zajištěno, že se nástroj dostane ke svým uživatelům v krátkém čase, vzhledem k napojení hlavních distribucí na tento repozitář.

Pokud mohu zhodnotit výsledky této práce, veřím že jsem splnil body zadání. Podařilo se vytvořit fungující nástroj pro kontrolu a korekci médií naformátovaných souborovým systémem UDF pro GNU/Linux, čímž se stává první takovou implementací s otevřeným zdrojovým kódem. Pokud mé řešení porovnám s komerčně dostupnými variantami, svou funkcí předčí například nástroj \texttt{CHKDSK} z prostředí Microsoft Windows, který nedokáže poškozené médium opravit tak, aby jej byl operační systém Microsoft Windows používat, zatímco mnou vytvořený nástroj toto dokáže. Nástroj pro kontrolu souborových systému v prostředí Apple macOS sice provádí opravy, ale neprovádí žádné korekce, tudíž nelze jeho činnost srovnat s činností mého nástroje.
