\chapter{Závěr}
Cílem práce bylo vytvořit nástroje pro diagnostiku integrity dat souborového systému pro OS Linux pro který v současnosti tato podpora neexistuje.\\
Během rešerše souborových systémů používaných v OS Linux bylo zjištěno, že neexistuje nástroj \texttt{fsck} pro souborový systém UDF. Proto jsem zvolil tento souborový systém pro další práci. Tomu předcházela analýza souborového systému z hlediska jeho funkce, použití a přístupu k detekci a korekci chyb.\\
Na tomto základě byly definovány detekovatelné a opravitelné chyby kterými má smysl se zabývat. V zásadě jde o chyby v metadatech souborového systému, protože data samotná jsou chráněna kontrolními a opravnými mechanismy nosného média. Metadata jsou na úrovni UDF chráněna kombinací kontroly polohy na médiu, kontrolních součtů a CRC. Další implementovaný a využitelný mechanismus je částečná redundance některých metadat. Poslední a klíčovou vlastností je značení neukončeného zápisu ať už v metadatech nebo pomocí ostatních indikátorů, které jsou popsány v kapitole \ref{ch:definice-chyb}.\\
Z tohoto vyplývá, že není možné opravovat chyby, které vznikly poruchou média ale spíše chyby provozní jako například nekorektní odpojení média během zápisu nebo porucha ovladače souborového systému.\\
Práce na nástroji \texttt{udffsck} byla započata v rámci existujícího balíčku \texttt{udftools}, který je součástí všech hlavních distribucí GNU/Linux. Tento balíček byl připravený pro existenici tohot nástroje ale práce na něm nikdy nezačala. Nástroj \texttt{udffsck} byl vytvořen v jazyce C podle standardu C99 s využitím funkcí poskytnutých operačním systémem \mbox{GNU/Linux}. Nástro je přeložitelný s minimálními nároky na hostitelský systém stejně jako zbytek balíčku.\\
Nástroj \texttt{udffsck} sloučuje jak detekční, tak korekční funkci. Nástroj dokáže detekovat a opravovat všechny definované chyby na médiích naformátovaných souborovým systémem UDF až do verze standardu 2.01.\\
Toto bylo otestováno vůči vybrané množině testovacích dat, která zachycuje nečastěji se vyskytující verze UDF. Způsoby tvorby a použití testovacích dat jsou popsány v kapitole \ref{ch:result}.\\
Výsledky práce byly publikovány v komunitě GNU začleněním výsledného nástroje \texttt{udffskc} do existujícího balíčku \texttt{udftools}. Tímto je zajištěno, že se nástroj dostane k uživatelům vzhledem k napojení hlavních distribujcí na tentor repozitář.\\

