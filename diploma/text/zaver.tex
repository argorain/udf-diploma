\chapter{Závěr}
Cílem práce bylo vytvořit nástroje pro diagnostiku integrity dat souborového systému pro OS Linux pro který v současnosti tato podpora neexistuje.\\
Během rešerše souborových systémů použávaných v OS Linux bylo zjištěno, že neexistuje nástroj \texttt{fsck} pro souborový systém UDF. Proto jsem zvolil tento souborový systém pro další práci.\\
Byly definovány detekovatelné a opravitelné chyby kterými má smysl se zabývat. V zásadě jde jen o chyby v metadatech souborového systému, prtože ty jsou chráněny kombinací kontrolních součtů a CRC. Další implementovaný a využitelný mechanismus je částečná redundance některých metadat. Poslední využitelnou vlastností je značení neukončeného zápisu. Samotná data nemají implementovaný žádný kontrolní mechanismus na úrovni souborového systému.\\
Z tohoto vyplývá, že není možné opravovat chyby vzniknutvší chybou média ale spíše chyby provozní jako například nekorektní odpojení média během zápisu.\\
Práce na nástroji \texttt{udffsck} byla započata v rámci exstujícího balíčku \texttt{udftools}, který je součástí všech hlavních distibucí GNU/Linux. Tento balíček byl připravený pro tento nástroj ale práce na něm nikdy nezačala. Nástroj je přeložitelný s minimálními závislostmi na hostitelský systém stejně jako zbytek balíčku.\\
Aktuální stav implementace dokáže načítat UDF souborové systémy do verze 2.01 a provést nad nimi kontrolu kontrolních součtů a CRC nad všemi metadaty souborového systému i dat. Tyto výsledky jsou vypisovány na standardní výstup.\\
Další práce bude spočívat v rozšíření tohoto nástroje o korekční schopnost, kdy dokáže nalezené chyby automaticky opravovat a ukládat zpět, pokud to bude možné. V ideálním případě dojde i k rozšíření maximální verze kontrolovaného média. Zbytek nástrojů v tuto chvíli podporuje pouze revizi 2.01, která odpovídá rozšíření pro DVD a od ní je odvezena i implementace \texttt{udfsck}. Rozšíření pro Blu-ray revizí 2.60 není v současnosti v balíčku \texttt{udftools} začleněno, tudíž i vývoj pro tuto revizi je komplikovanější. 
