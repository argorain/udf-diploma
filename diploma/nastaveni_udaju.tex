%% Název práce:
%  První parametr je název v originálním jazyce,
%  druhý je překlad v angličtině nebo češtině (pokud je originální jazyk angličtina)
\nazev{Nástroje pro diagnostiku integrity souborového systému v OS Linux}{Diagnostic tools for OS Linux file system}

%% Jméno a příjmení autora ve tvaru
%  [tituly před jménem]{Křestní}{Příjmení}[tituly za jménem]
\autor[Bc.]{Vojtěch}{Vladyka}

%% Jméno a příjmení vedoucího včetně titulů
%  [tituly před jménem]{Křestní}{Příjmení}[tituly za jménem]
% Pokud vedoucí nemá titul za jménem, smažte celý řetězec '[...]'
\vedouci[Ing.]{Petr}{Petyovský}

%% Jméno a příjmení oponenta včetně titulů
%  [tituly před jménem]{Křestní}{Příjmení}[tituly za jménem]
% Pokud nemá titul za jménem, smažte celý řetězec '[...]'
% Uplatní se pouze v prezentaci k obhajobě
\oponent[doc.\ Mgr.]{Křestní}{Příjmení}[Ph.D.]

%% Označení oboru studia
% První parametr je obor v originálním jazyce,
% druhý parametr je překlad v angličtině nebo češtině
\oborstudia{Kybernetika, automatizace a měření}{Cybernetics, Control and Measurements}

%% Označení ústavu
% První parametr je název ústavu v originálním jazyce,
% druhý parametr je překlad v angličtině nebo češtině
\ustav{Ústav automatizace a měřicí techniky}{The Department of Control and Instrumentation} 

%% Rok obhajoby
\rok{2016}
\datum{1.\,1.\,1970} % Uplatní se pouze v prezentaci k obhajobě

%% Místo obhajoby
% Na titulních stránkách bude automaticky vysázeno VELKÝMI písmeny
\misto{Brno}

%% Abstrakt
\abstrakt{Tato práce se zabývá problematikou návrhu nástroje pro detekci chyb na souborovém systému UDF pro OS Linux.}
{Aim of this work is creation of error detection tool for UDF filesystem for OS Linux.}

%% Klíčová slova
\klicovaslova{UDF, fsck, Linux, chyba, CRC, kontrolní součet}%
	         {UDF, fsck, Linux, error, CRC, checksum}

%% Poděkování
\podekovanitext{Rád bych poděkoval vedoucímu semestrální práce panu Ing.~Petru Petyovskému\ za odborné vedení, konzultace, trpělivost a podnětné návrhy k~práci.}
